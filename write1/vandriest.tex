\documentclass[a4paper,12pt]{article}

\usepackage{amsmath}
\usepackage{amssymb}
\usepackage{graphicx}
\usepackage{geometry}

\newcommand{\p}{\partial} 

\title{van Driest's model for wall scaling} 



\begin{document}

We identify the total shear stress $\tau$ as
\begin{equation}
  \tau = \underbrace{\mu \frac{\p \overline{u}}{\p y}}_\text{viscous stress} - \underbrace{\rho \overline{u^\prime v^\prime}\vphantom{\frac{\p \overline{u}}{\p y}}}_\text{Reynolds' stress}, 
\end{equation}
where $y$ is distance from the wall, $\mu$ dynamic viscosity ($\nu=\mu/\rho$) and $\rho$ is density. 
Next, observe that the amplitude of motion due to oscillation of an infinite plate decays as $\exp\left[-y/A\right]$. Hence, we use for the damping of fluid oscillation due to a fixed wall, the model
\begin{equation}
  1- e^{-\tilde{y}},
\end{equation}
where $\tilde{y}=y/A$. Now, we express the stress, according to Prandtl
\begin{subequations} 
\begin{eqnarray}
  \frac{\tau}{\rho} &=& \nu \frac{\p \overline{u}}{\p y} +  r \sqrt{\overline{u^{\prime2}}} \sqrt{\overline{v^{\prime2}}}  \\
       &=& \nu \frac{\p \overline{u}}{\p y} + r l_1 l_2 \left(\frac{\p \overline{u}}{\p y} \right)^2 \\
       &=& \nu \frac{\p \overline{u}}{\p y} + \kappa^2 y^2 \left(\frac{\p \overline{u}}{\p y} \right)^2, 
\end{eqnarray}
\end{subequations}
where $\kappa$ is the von-K\'arm\'an constant and $l$ Prandtl's mixing length. This model is well-known to hold appropriately in fully developed turbulent flow.

\par

Near a wall, however, the turbulence is not fully developed, but damped by the presence of that very wall such that the prefactor $1-e^{-\tilde{y}}$ can be taken into account in Reynolds' stres term:
\begin{eqnarray}
  \label{eqn:stress_model_dimensional}
  \frac{\tau}{\rho} &=& \nu \frac{\p \overline{u}}{\p y} + \kappa^2 l^2 \left(1- e^{-\tilde{y}}\right)^2\left(\frac{\p \overline{u}}{\p y}\right)^2
\end{eqnarray} 
Non-dimensionalize Eq.~(\ref{eqn:stress_model_dimensional}) using $u_\star=\sqrt{\tau_\text{wall}/{\rho}}$ and the wall unit $y_+=\nu/\sqrt{\tau_\text{wall}/\rho}$:
\begin{eqnarray}
  \tau^+ \left( = \frac{\tau}{\tau_\text{wall}} \right) = \frac{\p u^+}{\p y^+} + \kappa^2 y^{+2} \left(1- e^{-\widetilde{y^+}}\right)^2 \left(\frac{\p u^+}{\p y^+} \right)^2
\end{eqnarray}

In the constant-flux layer, it is $\tau=\tau_\text{Wall}$, such that
\begin{eqnarray}
  0 &=&  \frac{\p u^+}{\p y^+} + \kappa^2 y^{+2} \left(1- e^{-\widetilde{y^+}}\right)^2 \left(\frac{\p u^+}{\p y^+} \right)^2 - 1 \\ 
  0 &=&  \left(\frac{\p u^+}{\p y} \right)^2 + \frac{1}{\kappa^2 y^{+2}\left(1-e^{-\widetilde{y^+}}\right)^2} \left(\frac{\p u^+}{\p y^+} -1\right)
\end{eqnarray}
and we solve for $\p_{y^+}u^+$ to obtain 
\begin{eqnarray} 
  \frac{\p u^+}{\p y^+} = \frac{2}{1+\sqrt{1+4 \kappa^2 y^{+2}\left(1-e^{-\widetilde{y^+}}\right)}} 
\end{eqnarray}

\end{document}
