\documentclass[a4paper,1pt]{amsart}
\usepackage{amssymb}
\usepackage{amsmath}
\newcommand{\RE}{\mathbf{Re}}

\date{\today}
\title{Note on Boundary condition and choice of grid} 
\begin{document}
\noindent By choice of the grid, it is for the first grid point ($z_1$)
\[ z_1 = c_\delta \delta \Rightarrow z_1^{-}=c_\delta.\]
The height argument of the log-law is, however, an inner height, and we need to use $z^+$.
Using the relation $z^-=z^+/\RE_\tau$, we get 
\[ z_{1}^{+} = c_\delta \RE_\tau\] 
for the height of the first grid node in inner units.
%
This is a pure consequence of the construction of the grid under changing $\RE$ (namely using a constant number of grid points per $\delta$.
%
A similar relation holds for the rough equivalent:
\[ \frac{z_1}{z_0} = \frac{c_\delta \delta}{z_0} = c_\delta \frac{u_\star}{f} \frac{u_\star}{c_\nu \nu} = \frac{c_\delta}{c_\nu} \RE_\tau.\]
This underlines the role of $c_\nu$ in linking the classic inner ('+') scale and the rough inner unit ($z_0$), it is namely
\[ \frac{z^+}{z/z_0} = c_\nu.\]
First, this shows that--apart from iny variation in $c_\nu$--the scaling with $z_0$ is the classic inner ('+') scaling.
%
Second, the small variation of $c_\nu$ (estimated/tuned against the DNS results[!!!])
across the Reynolds number (less than the alread weak variation in $u_\star$
and orders of magnitudes less than the variation in $Re_\tau$) shows that the analogy between aerodynamically rough
and smooth flows holds quite nice, but not exact.
%
In that sense, the collapse of profile for the vertical scale $z/z_0$ is to be understood as an 'approximate' inner scaling. 
%
It remains to be seen whether the small variation of $c_\nu$ with $\RE$ is due to technical details of the LES formulation under changing
scale separation and different action of the SGS scheme, or whether these are an actual physical property of the flow.
%
It may also be that by changing $c_\nu$ we capture the inadverted effect of $z_1$ coming too close to the surface for
some configurations. 
%
\par
%
With regards to the velocity boundary condition, we now see that
\[U_1=U(z_1) = \frac{u_\star}{\kappa} \ln \frac{z_1}{z_0}  = \frac{u_\star}{\delta} \ln\RE_\tau - \ln\frac{c_\delta}{c_\nu}\]
where the first term ($\ln\RE_\tau$) captures the dependency of the grid on the Reynolds number and the second term reflects the tuning parameter
used to get a correct $u_\star$ in the LES.
%
\par
%




\end{document} 
